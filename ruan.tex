\documentclass[a4paper,10pt]{article}
\usepackage[brazil]{babel}
\usepackage[utf8]{inputenc}
\usepackage[pdftex]{hyperref}
\usepackage[utf8]{inputenc}

\title{{\textbf{Desenvolvimento de software para análise da tribo Enawene-nawe}}}
\author{\LARGE Ruan de Menezes Costa}
\date{}

\pdfinfo{%
  /Title    (Desenvolvimento de software para análise da tribo Enawene-nawe)
  /Author   (Ruan de Menezes Costa)
  /Creator  (Ruan de Menezes Costa)
  /Producer (Ruan de Menezes Costa)
  /Subject  ()
  /Keywords (Software Enawene-nawe grafos visualização consultas)
}

\begin{document}
\maketitle
\begin{center}

  {\scshape \huge Projeto de Iniciação Científica
	\\apresentado ao
\\Instituto de Matemática e Estatística
	     \\da
    \\Universidade de São Paulo
	    \\para
      \\obtenção de bolsa
	    \\do
\\Conselho Nacional de Desenvolvimento 
     \\Científico e Tecnológico
	   \\(CNPq)}
	   \bigskip
	   \bigskip
	   \bigskip
  \\{\LARGE Orientador: Carlos Eduardo Ferreira}
  
\end{center}
\newpage

\section{Introdução}
Um problema muito interessante em Antropologia é o estudo do
parentesco. É comumente aceito na área que as relações de parentesco
não se baseiam apenas em aspectos biológicos, uma vez que dependendo
da cultura observada elas variam enormemente. os primeiros estudos da
área já têm mais de 150 anos \cite{Morgan}. Uma interessante
observação sobre parentesco é a chamada \textbf{teoria da aliança}
desenvolvida pelo antropológo Claude Lévi-Strauss
\cite{LeviStrauss}. Nela o antropólogo propõe que os casamentos
ocorrem na sociedade como objetos de troca entre as famílias, a fim de
que estas possam escapar do tabu do incesto, presente em todas as
sociedades conhecidas. Assim, \textbf{regras de troca} surgem como
naturais, ou, até mesmo, prescritas na sociedade, como, por exemplo, a
chamada \textbf{troca de irmãos}, em que dois irmãos casam-se com duas
irmãs. 

O estudo destes objetos nas genealogias deu origem ao uso de grafos
para representar o parentesco \cite{Ore}. Neles os vértices são os
membros da comunidade e os arcos podem representar relação de filiação
ou casamento. Problemas em algoritmos em grafos (como
busca de ancestral comum mais próximo, junções, etc) têm interessantes
aplicações no estudo do parentesco, como vemos em \cite{Marcio,
  Alvaro}. 

Do interesse dos antropólogos Márcio Ferreira da Silva e João Dal Poz
Neto \cite{Marcio}
em estudar as relações de parentesco na tribo Enawene-nawe, 
surgiu a necessidade da criação de um software para facilitar esta
tarefa. Várias tarefas realizadas pelos antropólogos em seus estudos
de campo podem ser facilitadas com o apoio computacional. Um exemplo
disso é a consulta sobre as relações de consanguinidade de duas
pessoas. Essas relações variam de sociedade para sociedade, e também
no tempo, de acordo  com acontecimentos como casamentos e mortes. O
estudo desta nomenclatura e sua evolução no tempo pode ser feito com o
uso de algoritmos em grafos, e é um dos principais objetos de estudo
deste projeto. 
\section{Objetivos}
O projeto tem por objetivo a criação de um software para a visualização e análise de dados extraídos da tribo Enawene-nawe.
O usuário poderá verificar, por exemplo, qual é a relação de parentesco entre dois indíviduos dentro da comunidade, além de prover fácil acesso às suas características.
Também será possível escolher um ano como critério de seleção (para se saber como era a estrutura da tribo em determinado ano), dessa maneira, 
será possível analisar a evolução das relações de parentesco dos
habitantes da comunidade ao longo do tempo. Também é de interesse expor o digrafo de maneira apropriada, tornando sua análise mais fácil.

\section{Metodologia}
O software será desenvolvido na linguagem de programação Java. 
A entrada do programa é um arquivo texto que descreve as características dos indivíduos
(número de identificação, pai, mãe, data de nascimento, data de morte, casamento).
A comunidade será modelada por um digrafo, onde cada indivíduo é um nó e as relações de parentesco e matrimônio são arcos.
Será usada a biblioteca GraphStream \cite{graphstream} para a manipulação e exibição do digrafo.
Para a visuallização, será tomado como base o framework de Sugyama \cite{sugiyama02}, que consiste em 3 passos para otimizar o layout de um grafo dirigido acíclico.
\begin{enumerate}
  \item Separação dos nós do digrafo por camadas. Isto pode ser conseguido por meio de vários algoritmos, como por exemplo, Coffman-Graham \cite{coffman} e Longest-path.
  \item Organização dos nós dentro da camada. Etapa necessária para diminuir o número de cruzamento de arcos. Será feita por meio de heurísticas como o método da mediana ou do baricentro.
  \item Designação das coordenadas para os nós. Além da ordem dos
    vértices, a posição dentro da camada também pode prejudicar o
    layout do digrafo. Será usado o método de Brandes e Köpf
    \cite{brandes02}, entre outros. 
\end{enumerate}
 Para as consultas, será feito um pré-processamento utilizando-se algoritmos elementares de dígrafos, como busca em profundidade, para a determinar a relação dos indíviduos dois a dois.
\begin{thebibliography}{9}

 \bibitem{brandes02}
  Ulrik Brandes, and Boris Köpf. \emph{Fast and simple horizontal coordinate assignment}.
  Graph Drawing. Springer Berlin Heidelberg, 2002.
  
  \bibitem{coffman}
   E.G. Coffman Jr, and Ronald L. Graham. 
  \emph{Optimal scheduling for two-processor systems}. 
  Acta Informatica 1.3 (1972): 200-213.

\bibitem{Alvaro}
Alvaro Junio Pereira Franco, \emph{Algorithms for junctions in acycli
  digraphs and an application in Antropology}, Ph.D. Thesis,
University of São Paulo. 

 \bibitem{graphstream}
  GraphStream,
  2010-2013.
  URL:\url{http://graphstream-project.org/}

\bibitem{LeviStrauss}
Claude Lévi-Strauss, 
\emph{Les structures élémentaires de la parenté}, Mouton, Paris, 1967.

 \bibitem{Morgan}
 Lewis Henry Morgan, 
\emph{Systems of consanguinity and affinity of the human family},
University of Nebrasca Press, 1870. 

\bibitem{Ore}
Oystein Ore, 
\emph{Sex in graphs}, Proceedings of the American Mathematical
Society, 533--539, 1960. 

\bibitem{Marcio}
Marcio Ferreira da Silva and João dal Poz, 
\emph{Maqpar: a homemade tool for the study of kinship networks},
Vibrant 6(2), 29--51, 2009. 

 \bibitem{sugiyama02}
 Kozo Sugiyama, Shojiro Tagawa, and Mitsuhiko Toda.
  \emph{Methods for visual understanding of hierarchical system structures}.
  Systems, Man and Cybernetics,
  IEEE Transactions on 11.2 (1981): 109-125.

\end{thebibliography}

\end{document}
